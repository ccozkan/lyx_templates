%% LyX 2.2.2 created this file.  For more info, see http://www.lyx.org/.
%% Do not edit unless you really know what you are doing.
\documentclass[english]{article}
\usepackage[T1]{fontenc}
\usepackage[latin9]{inputenc}
\usepackage{geometry}
\geometry{verbose,tmargin=3cm,bmargin=2cm,lmargin=2cm,rmargin=2cm}
\setlength{\parskip}{\medskipamount}
\setlength{\parindent}{0pt}
\usepackage{float}
\usepackage{units}
\usepackage{url}
\usepackage{graphicx}

\makeatletter
%%%%%%%%%%%%%%%%%%%%%%%%%%%%%% User specified LaTeX commands.

\usepackage{fancyhdr}
\pagestyle{fancy}
\lhead{ME 477 FEA, Fall 2015}
\chead{C. �a\u{g}r\i{} �zkan}
\rhead{FEM Project}
\rfoot{Yeditepe University}
\cfoot{\thepage}
\lfoot{Mechanical Engineering}
\usepackage{tikz}
\usepackage{pgfplots}
    % recommended as of Pgfplots 1.3:
\pgfplotsset{compat=newest}

\@ifundefined{showcaptionsetup}{}{%
 \PassOptionsToPackage{caption=false}{subfig}}
\usepackage{subfig}
\makeatother

\usepackage{babel}
\usepackage{listings}
\renewcommand{\lstlistingname}{Listing}

\begin{document}

\title{\textbf{\Huge{}FE Analysis of 2D Bridge Structure}\\
\textbf{\Huge{}}\\
\texttt{\huge{}me 477 fea project, fall 2015}}

\author{C. �a\u{g}r\i{} �zkan}

\date{05-I-2016}

\maketitle
\pagebreak

\section{Problem and Overview}

I choose the similar problem to example problem which can be found
at ME 477 section at unilica{[}dot{]}com. The problem to deal with
is how some arched bridge structure behaves under load which is applied
at its peak point by finite element method. FEAP is used for its FE
Analysis. In finite element truss method will be used.

The example geometry can be seen in Figure \ref{fig:3D-Model}. 

\begin{center}
\begin{figure}[H]
\begin{centering}
\includegraphics[width=0.6\textwidth]{Desktop/fempro/Screenshot_2016-01-03_01-19-08}
\par\end{centering}
\caption{3D Model\label{fig:3D-Model}}

\end{figure}
\par\end{center}

In these problem given parameters are $\alpha=72^{o}$ , $\hat{EOC=}\nicefrac{\alpha}{3}$
and $R=14m$. By knowing those properties the locations of points
can be determined easily. And in case of interest the element lengths
can be found from Law of Cosine(Eq. \ref{eq:cos})

\begin{equation}
a^{2}=b^{2}+c^{2}-2\:b\:c\:cos(\alpha)\label{eq:cos}
\end{equation}

Location points in Cartesian Coordinates are found using Octave. Finding
A and E point locations and |AE| length are demonstrated. 

\begin{lstlisting}[basicstyle={\normalsize\ttfamily},breaklines=true]
octave:1> A=[-14*cosd((180-72)/2);14*sind((180-72)/2)] 
A =
   -8.2290  
   11.3262
octave:2> E=[-14*cosd((180-72)/2+24);14*sind((180-72)/2+24)] 
E =
   -2.9108
   13.6941
octave:3> AE=sqrt((14^2+14^2-2*14*14*cosd(24)))
AE =  5.8215 
\end{lstlisting}

Structure is simplified as follows due to use the advantages of 2-D
analysis. Figure \ref{fig:Simplified-Model} shows the simplified
2D model.

\begin{center}
\begin{figure}[H]
\begin{centering}
\includegraphics[width=0.3\textheight]{Desktop/fempro/1}
\par\end{centering}
\caption{Simplified Model\label{fig:Simplified-Model}}

\end{figure}
\par\end{center}

As it can be seen from the figure \ref{fig:Simplified-Model} , this
design have 5 nodes and 4 elements. 

As a material A36 steel is chosen. Which has a Young's Modulus (E)
200 10\textsuperscript{9}{[}Pa{]}. Cross sectional area is chosen
to be 0.1 meter-squared and squared profile, also applied force chosen
to be 30kN to demonstrate heavy load on the bridge\footnote{Maximum truck load is restricted with 27 metric tones by the law in
Republic of Turkey.}.

Also fixed points which are the beginning and the end of the bridge,
will be used for boundary conditions of the problem.

But with these force magnitude and material values, displacement values
came out to be enormous. In this point I crossed with lowering the
force or changing the design dilemma, and I went with changing the
design under same material and force properties to get practical results. 

As a modification 3 nodes and 9 elements are added. The coordinates
of the nodes and the structure can be seen on Figure \ref{fig:Design-v2}.
The modified design's nodes, elements and degree of freedoms are shown
in Figure \ref{fig:nedof}.

\begin{center}
\begin{figure}[H]
\begin{centering}
\subfloat[\label{fig:Design-v2}]{\begin{centering}
\includegraphics[width=0.3\textheight]{Desktop/fempro/2}
\par\end{centering}
}
\par\end{centering}
\begin{centering}
\subfloat[\label{fig:nedof}]{\begin{centering}
\includegraphics[width=0.6\textwidth]{Desktop/fempro/tar/3}
\par\end{centering}
}
\par\end{centering}
\caption{Design v2}

\end{figure}
\par\end{center}

\pagebreak

\section{Pre-Process}

To solve the problem input file is created in favorite text editor.
Even though odd look of it, it just contains the info about its nodes
coordinates, elements that are created between these nodes and material
properties of those elements, applied force magnitude and its whereabouts,
boundary conditions and lastly what we hope to get from analysis.
The input file can be seen below.

\begin{lstlisting}[numbers=left,numberstyle={\footnotesize},basicstyle={\normalsize\ttfamily},breaklines=true]
FEAP * *  2-D Bridge Structure
  0 0 0 2 2 2


MATErial,1
  TRUSS
    ELAStic ISOTropic 200e9
    CROSs SECTion     0.1

COORdinates
  1 0 -8.2 11.3
  2 0 -2.9 13.7
  3 0 0.0 14.0
  4 0 2.9 13.7
  5 0 8.2 11.3
  6 0 -4.1 11.3
  7 0 0.0 11.3
  8 0 4.1 11.3 

ELEMents
  1 0 1 1 2
  2 0 1 2 3
  3 0 1 3 4
  4 0 1 4 5
  5 0 1 1 6
  6 0 1 6 7
  7 0 1 7 8
  8 0 1 8 5
  9 0 1 6 2
  10 0 1 2 7
  11 0 1 7 3
  12 0 1 3 8
  13 0 1 8 4

BOUNdary restraints
  1 0 1 1
  5 0 1 1

FORCe
  3 0 0.0 -30000.0

END
TIE

BATCH
  TANG,,1
  DISPlacement ALL
  STRESs       ALL
  REACtion     ALL
  PLOT  MESH
  PLOT LOAD 1,-1
  PLOT  CONT   1
END


INTEractive
STOP
\end{lstlisting}

According to these input file mesh created by FEAP can be seen on
Figure\ref{fig:Mesh}.

\begin{center}
\begin{figure}[H]
\begin{centering}
\includegraphics[width=0.6\textwidth]{\string"Desktop/fempro/tar/truss dosyasi/mesh\string".eps}
\par\end{centering}
\caption{Mesh\label{fig:Mesh}}

\end{figure}
\par\end{center}

\pagebreak

\section{Post-process \& Conclusion}

After the solution is run, output file file and colorized picture
for displacements are created {[}Figure \ref{fig:Colorized-Displacement}{]}.

\begin{center}
\begin{figure}[H]
\begin{centering}
\includegraphics[width=0.8\textwidth]{\string"Desktop/fempro/tar/truss dosyasi/trus\string".eps}
\par\end{centering}
\caption{Colorized Displacement\label{fig:Colorized-Displacement}}

\end{figure}
\par\end{center}

Since output file is 228 lines long; only the interested sections
of it is posted below. Full version can be found at \url{http://pastebin.com/tduNBxsi}.

Following section of output file contains information about nodal
displacements. If results would be discussed they are practical since
their very small values. 

\begin{lstlisting}[basicstyle={\small\ttfamily}]
 FEAP * * 2-D Bridge Structure                                                 

  N o d a l   D i s p l a c e m e n t s     Time       0.00000E+00
                                            Prop. Ld.  1.00000E+00

    Node     1 Coord     2 Coord     1 Displ     2 Displ
       1 -8.2000E+00  1.1300E+01  0.0000E+00  0.0000E+00
       2 -2.9000E+00  1.3700E+01  4.0536E-06 -3.4595E-05
       3  0.0000E+00  1.4000E+01 -1.3559E-06 -4.7185E-05
       4  2.9000E+00  1.3700E+01 -4.0293E-06 -3.4542E-05
       5  8.2000E+00  1.1300E+01  0.0000E+00  0.0000E+00
       6 -4.1000E+00  1.1300E+01 -6.3707E-07 -3.2250E-05
       7  0.0000E+00  1.1300E+01 -1.2741E-06 -4.5796E-05
       8  4.1000E+00  1.1300E+01  6.3707E-07 -3.4255E-05
 *Command   3 * stre ALL            v:   0.00       0.00       0.00    
                                                           t=     0.01     0.00
\end{lstlisting}

Following section of output file contains information about force
and strain. 

\begin{lstlisting}[basicstyle={\normalsize\ttfamily}]
 FEAP * * 2-D Bridge Structure                                                   

         Truss Element

 Elmt Matl    1-coord    2-coord    3-coord     Force         Strain          Flux
    1    1 -5.550E+00  1.250E+01  0.000E+00  -3.63630E+04  -1.81815E-06   0.00000E+00
    2    1 -1.450E+00  1.385E+01  0.000E+00  -4.57987E+04  -2.28993E-06   0.00000E+00
    3    1  1.450E+00  1.385E+01  0.000E+00  -2.71669E+04  -1.35835E-06   0.00000E+00
    4    1  5.550E+00  1.250E+01  0.000E+00  -3.63630E+04  -1.81815E-06   0.00000E+00
    5    1 -6.150E+00  1.130E+01  0.000E+00  -3.10764E+03  -1.55382E-07   0.00000E+00
    6    1 -2.050E+00  1.130E+01  0.000E+00  -3.10764E+03  -1.55382E-07   0.00000E+00
    7    1  2.050E+00  1.130E+01  0.000E+00   9.32292E+03   4.66146E-07   0.00000E+00
    8    1  6.150E+00  1.130E+01  0.000E+00  -3.10764E+03  -1.55382E-07   0.00000E+00
    9    1 -3.500E+00  1.250E+01  0.000E+00   6.35275E-11   3.17637E-21   0.00000E+00
   10    1 -1.450E+00  1.250E+01  0.000E+00   1.61353E+04   8.06766E-07   0.00000E+00
   11    1  0.000E+00  1.265E+01  0.000E+00  -1.02874E+04  -5.14368E-07   0.00000E+00
   12    1  2.050E+00  1.265E+01  0.000E+00  -2.21905E+04  -1.10952E-06   0.00000E+00
   13    1  3.500E+00  1.250E+01  0.000E+00   1.36451E+04   6.82255E-07   0.00000E+00
 *Command   4 * reac ALL            v:   0.00       0.00       0.00    
                                                           t=     0.01     0.00
\end{lstlisting}

It should be noted that force can lead to stress with easy algebra.

\[
\sigma=\frac{F}{A}
\]
 Basic manipulation gives $\sigma$ values as follows:

\begin{lstlisting}[basicstyle={\normalsize\ttfamily}]
octave:2> Force/0.1
ans =

  -3.6400e+05
  -4.5800e+05
  -2.7200e+05
  -3.6400e+04
  -3.1100e+04
  -3.1100e+04
   9.3200e+04
  -3.1100e+04
   6.3500e-10
   1.6100e+05
  -1.0300e+05
  -2.2000e+05
   1.3600e+05
\end{lstlisting}

Which is safe since yield strength of A 36 steel is 260 MPa.

Following section of output file contains information about nodal
reactions. 

\begin{lstlisting}
 FEAP * * 2-D Bridge Structure                                                 

  N o d a l    R e a c t i o n s

   Node       1 dof       2 dof
      1  3.6233E+04  1.5000E+04
      2 -4.9113E-11  6.0027E-11
      3  2.9104E-11 -3.0000E+04
      4 -5.0932E-11  1.2733E-11
      5 -3.6233E+04  1.5000E+04
      6 -4.1598E-11 -5.6821E-11
      7  7.2760E-12 -4.0018E-11
      8  1.2733E-11  3.0923E-11

 Pr.Sum -7.0372E-12  1.3870E-12
   Sum  -7.0372E-12  1.3870E-12
  |Sum|  7.2465E+04  6.0000E+04
   Energy: Displacements * Reactions =     1.415548070137809E+00
\end{lstlisting}

Very small reactions except node 1 and 5 which are bridge's beginning
and end points. It can be seen that nodal reactions are very small
except node 1 and 5 which are bridge's beginning and end points and
therefore fixed.

In conclusion modified bridge structure definitely can be considered
safe since displacement, stress and node reactions are allowable.
I choose FEAP over Pyfem for the analysis for practical reasons. FEAP
was good enough to gave those results without too much struggle; but
I would like to deal with Pyfem in future. 
\end{document}
